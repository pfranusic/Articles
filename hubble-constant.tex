%% hubble-constant.tex

\documentclass{report}
\pagestyle{empty}

% Packages
\usepackage{verbatim}
\usepackage{graphicx}
\usepackage{latexsym}
\usepackage{amssymb}
\usepackage{amsmath}
\usepackage[usenames]{color}

% Commands
\newcommand{\xu}[1]{\;\mathrm{\scriptstyle{#1}}}
\newcommand{\xw}[2]{\;\textstyle{\frac{\mathrm{#1}}{\mathrm{#2}}}}

% Easy-vision mode
\pagecolor{black}
\color{green}

% Use European-style paragraphs.
% IMPORTANT: \begin{document} must follow for this to work.
\setlength{\parindent}{0pt} 
\setlength{\parskip}{1.3ex} 
\begin{document}

\begin{center}
\textbf{\Huge The Hubble Constant} \\
\vspace{2ex}
\textrm{\Large Peter Franu\v si\'c} \\
\end{center}
\vspace{0.5in}

Space expands.
The astronomer Edwin Hubble discovered that most galaxies appear to be receding from us,
and that the apparant velocity of a galaxy is directly proportional to its distance from us.
This phenomena is explained by the expansion of the space between the galaxies.

Space expands at a constant rate.
The rate of expansion is currently estimated to be about 
70.4 kilometers per second (km/s) per megaparsec (Mpc).
This rate is represented by the Hubble constant $H_0$.
\begin{equation*}
  H_0 = 70.4 \; \xw{km/s}{Mpc}
\end{equation*}

We can use the Hubble constant $H_0$ to compute the relative velocity $v$ 
of two points in space that are separated by a distance of $d$.
The formula is $  v = H_0 \times d$.
This means that two points in space, separated by a distance of 1 Mpc,
will move away from each other at a velocity of 70.4 km/s.
Another two points, separated by a distance of 10 Mpc,
will move away from each other at a velocity of 704 km/s.
The greater the distance, the greater the velocity.

But there's a problem here.
70.4 km/s is about 158 thousand miles per hour.
And 1 Mpc is about 31 million trillion kilometers.
Velocities and distances such as these are too huge to comprehend.
We need metrics that the average person is familiar with.

Let the velocity be in micrometers per gigasecond (um/Gs).
And let the unit of distance be picoparsecs (ppc).
A gigasecond is about 32 years, and picoparsec is about 19 miles.
So now, what's $H_0$ in terms of these units?
\begin{eqnarray*}
  H_0 &=& \frac{\Big( 70.4 \xw{km}{s}  \Big)
                \Big( 10^9 \xw{um}{km} \Big)
                \Big( 10^9 \xw{s}{Gs}  \Big)}
               {\Big( 1.0 \xu{Mpc}  \Big)
		\Big( 10^{18} \xw{ppc}{Mpc} \Big)} \\
      &=& 70.4 \; \xw{um/Gs}{ppc}
\end{eqnarray*}

70 micrometers is the average diameter of a human hair.
So two points 19 miles apart will move away from each other
at the rate of a hair width every 32 years.
This is comprehensible.  \emph{I.e.}, it's very, very slow.
Space expands very, very slowly.  But space is also immense.
At very great distances (\emph{e.g.} one billion light-years for quasars),
relative velocities approach the speed of light.



\end{document}

