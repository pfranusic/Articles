%% hubble-constant.tex

\documentclass{report}
\pagestyle{empty}

% Packages
\usepackage{verbatim}
\usepackage{graphicx}
\usepackage{latexsym}
\usepackage{amssymb}
\usepackage{amsmath}
\usepackage[usenames]{color}

% Commands
\newcommand{\grad}{\boldsymbol{\nabla}}
\newcommand{\bv}[1]{\boldsymbol{#1}}
\newcommand{\dfdx}[2]{\frac{\partial #1}{\partial #2}}

% Easy-vision mode
\pagecolor{black}
\color{green}

% Use European-style paragraphs.
% IMPORTANT: \begin{document} must follow for this to work.
\setlength{\parindent}{0pt} 
\setlength{\parskip}{1.3ex} 
\begin{document}

\begin{center}
\textbf{\Huge The Hubble Constant} \\
\vspace{2ex}
\textrm{\Large Peter Franu\v si\'c} \\
\end{center}
\vspace{0.5in}

Space expands.
The astronomer Edwin Hubble discovered that the galaxies appear to be receding from us,
and that the apparant velocity of a galaxy is directly proportional to its distance from us.
This phenomena is explained by the expansion of space.

Space expands at a constant rate.
The rate of expansion is currently estimated to be about 
70.4 kilometers per second (km/s) per Megaparsec (Mpc).
This rate is represented by the Hubble constant $H_0$.
\begin{equation*}
  H_0 = 70.4 \; \textstyle{\frac{\mathrm{km/s}}{\mathrm{Mpc}}}
\end{equation*}

We can use the Hubble constant $H_0$ to compute the relative velocity $v$ 
of two points in space that are separated by a distance of $d$.
The formula is $  v = H_0 \times d$.
This means that two points in space, separated by a distance of one Megaparsec,
will move away from each other at a velocity of 70.4 kilometers per second.
Another two points, separated by a distance of ten Megaparsecs,
will move away from each other at a velocity of 704 kilometers per second.
The greater the distance, the greater the velocity.

But there's a problem here.
A velocity of 70.4 kilometers per second is about 158 thousand miles per hour.
A distance of 1 Megaparsec is about 31 million trillion kilometers.
Velocities and distances such as these are huge.
In fact, they're beyond comprehension.
We need something that the average person can understand.

Let the velocity be in millimeters per century (mm/Cy).
And let the unit of distance be one hundred miles (Cmi).
So now, what is the Hubble constant $H_0$ in terms of these units?
%\begin{comment}
\begin{eqnarray*}
  H_0 &=& 70.4 \; \textstyle{\frac{\mathrm{km/s}}{\mathrm{Mpc}}} \\
      &=& \frac{(70.4 \; \textstyle{\frac{\mathrm{km}}{\mathrm{s}}})
                (10^6 \; \textstyle{\frac{\mathrm{mm}}{\mathrm{km}}})
                (3155760000 \; \textstyle{\frac{\mathrm{s}}{\mathrm{Cy}}})
                (160.9 \; \textstyle{\frac{\mathrm{km}}{\mathrm{Cmi}}})}
               {(1 \; \mathrm{\scriptstyle{Mpc}}) 
		(30.856776 \times 10^{18} \; \textstyle{\frac{\mathrm{km}}{\mathrm{Mpc}}})} \\
      &=& 1.16 \; \textstyle{\frac{\mathrm{mm/Cy}}{\mathrm{Cmi}}}
\end{eqnarray*}
%\end{comment}

This means that two points in space, separated by a distance of 100 miles,
will move away from each other at a velocity of about 1.16 mm per century.
This is comprehensible.  \emph{I.e.}, it's very, very slow.
Space expands very, very slowly.
But space is also immense.
At very great distances (\emph{e.g.} one billion light-years for quasars),
relative velocities approach the speed of light.

\end{document}

